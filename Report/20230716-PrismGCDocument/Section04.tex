\section{智能化IP设计}\label{sec:智能化IP设计}

系统的智能化设计主要体现在输入设备,WII Nunchuck手柄上。WII Nunchuck手柄是由任天堂公司推出的家用游戏主机配套外设。手柄包括2个按钮、2轴摇杆、3轴加速度传感器。手柄基于IIC协议工作,能方便的获取手柄上各个传感器的数据,具体包含:两个按钮(Z按钮和C按钮)是否按下、摇杆的横纵坐标、手柄的三轴加速度。

WII手柄的通讯采用IIC协议,包含SDA和SCL两路信号。由于初赛期间IIC协议的硬件实现未获成功,复赛期间,IIC协议改用软件实现,WII手柄直接挂载于GPIO。目前,WII手柄已经可以正常工作,通过手柄摇杆和体感方式控制游戏操作。

WII手柄的开发参照了一篇文档\footnote{\url{https://www.xarg.org/2016/12/using-a-wii-nunchuk-with-arduino/}},其中较重要的两项操作是
\begin{itemize}
    \item 手柄初始化:START、0x40、0x00、STOP。
    \item 手柄数据读取:START、0x00,、STOP、READ 6 byte。
\end{itemize}

\begin{Table}[WII手柄的数据排布]{cp{1cm}p{1cm}p{1cm}p{1cm}p{1cm}p{1cm}p{1cm}p{1cm}}
    <\mr{2}{Byte}&\mc{8}(c){Bit}\\
    &7\qquad&6\qquad&5\qquad&4\qquad&3\qquad&2\qquad&1\qquad&0\qquad\\>
    1&\mc{8}{\texttt{JX[7:0]}}\\ \hlinelig
    2&\mc{8}{\texttt{JY[7:0]}}\\ \hlinelig
    3&\mc{8}{\texttt{AX[9:2]}}\\ \hlinelig
    4&\mc{8}{\texttt{AY[9:2]}}\\ \hlinelig
    5&\mc{8}{\texttt{AZ[9:2]}}\\ \hlinelig
    6&
    \mc{2}(l|){\texttt{AX[1:0]}}&
    \mc{2}(l|){\texttt{AY[1:0]}}&
    \mc{2}(l|){\texttt{AZ[1:0]}}&
    \mc{1}(l|){$\neg$\texttt{BC}}&$\neg$\texttt{BZ}\\
\end{Table}

值得注意的是,根据网站的内容,若通过START、0x40、0x00、STOP初始化手柄,则后期读取到的数据\texttt{x}是经过加密的,需要使用\texttt{(x\^{}0x17)+0x17}解密,文档中亦有获取不加密数据的初始化方法。但是,通过对项目实际购买的WII手柄测试发现,依照START、0x40、0x00、STOP初始化手柄后得到的数据,实际上并没有被加密。


WII手柄传回的数据共$6$个字节,如\xref{tab:WII手柄的数据排布}所示。其中,\texttt{JX}、\texttt{JY}为摇杆$x,y$轴数据,\texttt{AX}、\texttt{AY}、\texttt{AZ}为$x,y,z$轴上的加速度传感器上的数据,\texttt{BC}、\texttt{BZ}则为按键C和Z的状态。

WII手柄的相关代码如下。\texttt{SCL}和\texttt{SDA}是IIC所需的两路信号,在项目中通过GPIO方式接入。\xref{code:IIC时序的实现}依照IIC时序实现了比特读写。\xref{code:IIC基本指令}中,\texttt{I2CSetSlave()}完成了三个任务:发送IIC的START信号、发送从设备的地址\texttt{slave}、设置读写状态位为\texttt{rw}(取值为\texttt{I2C\_WR}或\texttt{I2C\_RD})。\texttt{I2CWrite()}和\texttt{I2CRead()}进一步实现了字节的读写。\xref{code:WII手柄初始化和数据读取}则是依照前文叙述的方式,实现了手柄的初始化和数据读取功能,其中从设备地址根据阅读文档确定为0x52。\xref{code:WII手柄功能封装}则进一步将手柄读回的数据转换为按键操作,在按键Z未被按下时方向由摇杆控制,在按键Z按下时方向则由加速度传感器控制,即开启体感控制模式,另外,按键C的按下也被视为一个独立的按键操作。

\lstinputlisting[
    style       =   C,
    caption     =   {IIC时序的实现},
    label       =   {code:IIC时序的实现}
]{code/Code23.c}

\lstinputlisting[
    style       =   C,
    caption     =   {IIC基本指令},
    label       =   {code:IIC基本指令}
]{code/Code24.c}

\lstinputlisting[
    style       =   C,
    caption     =   {WII手柄初始化和数据读取},
    label       =   {code:WII手柄初始化和数据读取}
]{code/Code25.c}

\lstinputlisting[
    style       =   C,
    caption     =   {WII手柄功能封装},
    label       =   {code:WII手柄功能封装}
]{code/Code26.c}