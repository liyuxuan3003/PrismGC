\documentclass[compress,aspectratio=169]{ctexbeamer}

\usepackage{upgreek}

\usetheme{Luebeck}
\useoutertheme{miniframes}

\usecolortheme{seahorse}
\usecolortheme{orchid}

\usefonttheme{structuresmallcapsserif}

\AtBeginSection[]
{
	\begin{frame}
        \frametitle{目录}
		\tableofcontents[sectionstyle=show/hide,subsectionstyle=show/shaded/hide]
	\end{frame}
}
\AtBeginSubsection[]
{
	\begin{frame}
        \frametitle{目录}
		\tableofcontents[sectionstyle=show/hide,subsectionstyle=show/shaded/hide]
	\end{frame}
}

%设定中文字体
\setCJKmainfont[AutoFakeSlant=0.25]{Noto Serif CJK SC}      %谷歌衬线字体
\setCJKsansfont[AutoFakeSlant=0.25]{Noto Sans CJK SC}       %谷歌无衬线字体
\setCJKmonofont[AutoFakeBold,AutoFakeSlant=0.25]{FangSong}  %仿宋字体

\setlength{\parskip}{6pt}   %段落间距为6pt

\title{集创赛~安谋科技杯小组}
\subtitle{选题原因与初步规划}

\author{棱镜\textsc{PrisM}项目组}

\begin{document}

\begin{frame}
    \titlepage
\end{frame}

\begin{frame}
    \frametitle{组员信息}
    \begin{center}
        棱镜\textsc{PrisM}项目组:李宇轩、张岩、李铭杰
    \end{center}
\end{frame}

\begin{frame}
    \frametitle{总目录}
    \tableofcontents
\end{frame}

\section{选题原因}

\subsection{杯赛信息}

\begin{frame}[fragile]
    \frametitle{杯赛名称}
    \begin{center}
        \large
        数字与SoC设计赛道\qquad 
        安谋科技(Arm China)杯
    \end{center}\vspace{3ex}
    \begin{center}
        \url{http://univ.ciciec.com/nd.jsp?id=553#_jcp=1}
    \end{center}
\end{frame}

\begin{frame}
    \frametitle{杯赛题目}
    \begin{center}
        \large
        基于Arm处理器的智能游戏机设计
    \end{center}
\end{frame}

\subsection{杯赛间的对比}
\begin{frame}
    \frametitle{赛道特点}
    我们希望参与和FPGA有关的杯赛,故关注以下赛道
    \begin{itemize}
        \item \textbf{数字与SoC设计赛道}:在FPGA上可以使用CPU,硬软件代码相结合。
        \item \textbf{FPGA设计与应用赛道}:在FPGA上不能使用CPU,纯硬件代码,难度偏高。
        \item \textbf{芯片设计与应用本科赛道}:本科赛道,部分题目涉及FPGA。
    \end{itemize}
\end{frame}

\begin{frame}
    \frametitle{安谋科技杯的优势}
    \begin{enumerate}
        \item 硬件代码(Verilog)与软件代码(C)相结合,相对容易一些。
        \item 杯赛任务目标明确,思路清晰(相较雨骤杯、Robei杯)。
        \item 杯赛的设计目标是游戏机,竞争相对公平(相较景嘉微杯、芯原杯)
        \item 而且设计游戏机听上去也比较吸引人~:)
    \end{enumerate}
\end{frame}

\begin{frame}
    \frametitle{安谋科技杯的劣势}
    \begin{enumerate}
        \item 数字与Soc设计赛道并非纯本科生赛道,存在与研究生的竞争压力。
        \item 数字与Soc设计赛道的报名人数,是每年各赛道中最多的。
    \end{enumerate}
\end{frame}

\section{初步规划}

\subsection{关于FPGA芯片的选择}

\begin{frame}[fragile]
    \frametitle{芯片特点}
    杯赛允许使用的FPGA芯片有两款:安路科技的EG4S20、安路科技的PH1A60。

    安路科技是国内一家专注于FPGA业务的公司\footnote{\url{https://www.anlogic.com/}}
    \begin{itemize}
        \item EG4S20隶属于其SALEAGLE4(猎鹰)系列\footnote{\url{https://www.anlogic.com/product/fpga/saleagle/eg4}},$\text{LUTs}=19600$
        \item PH1A60隶属于其SALPHOENIX(凤凰)系列\footnote{\url{https://www.anlogic.com/product/fpga/phoenix/ph1a}},$\text{LUTs}=70848$
    \end{itemize}
\end{frame}

\begin{frame}
    \frametitle{芯片选择}
    尽管PH1A60芯片的性能远远高于EG4S20,但因为未知原因,目前在淘宝上很难搜到搭载PH1A60芯片的核心板,而现成的核心板是初期学习和测试所必须的。

    相反EG4S20的选择则丰富很多,核心板的价格普遍在250元左右。
    
    我们目前选中了硬木课堂设计的一款核心板\footnote{\url{https://item.taobao.com/item.htm?spm=a1z10.3-c-s.w4002-22468221883.9.6d741133IDlRsQ&id=666895953150}}
    \begin{enumerate}
        \item 硬木课堂似乎是专门生产这类适用于教学和竞赛用的电路板。
        \item 硬木课堂基于其核心板,提供了丰富的教程资料。\footnote{\url{https://www.yuque.com/yingmuketang/01/qha859}}
        \item 注意到有视频显示去年有参加安谋科技杯的队伍,使用的就是该款核心板。
        \item 硬木课堂就该核心板有丰富的配套扩展硬件,如显示屏、摄像头等。
    \end{enumerate}
\end{frame}

\begin{frame}
    \frametitle{现状}
    \begin{itemize}
        \item 计划现阶段先使用硬木课堂的EG4S20核心板进行学习和初步开发。
        \item 已经购置一块核心板,稍后会再购置两块,做到每个人手上都有。
        \item 已经在安路科技的官网上找到了EG4S20的芯片手册。
        \item 后期根据实际的硬件需要,决定是否更换性能更好的PH1A60芯片。
        \item 后期可能也会自行设计电路板。
    \end{itemize}
\end{frame}

\subsection{关于处理器}

\begin{frame}
    \frametitle{处理器}
    杯赛可以选用的处理器IP有两个:Cortex-M0、Cortex-M3。
    \begin{itemize}
        \item 两款芯片中,M3的性能应高于M0,但尚未调查具体的性能数据。
        \item 两款芯片均不支持内存管理,因此不可能在其上运行操作系统。
    \end{itemize}
\end{frame}

\subsection{关于软件}
\begin{frame}
    \frametitle{软件}
    杯赛涉及的软件主要包含两款
    \begin{enumerate}
        \item 安路科技的TangDynasty,用于编写硬件代码和FPGA烧写(正在研究\footnote{计划使用的版本是\texttt{TD\_RELEASE\_4.6.7\_64bit\_NL.msi}})\footnote{\url{https://www.anlogic.com/support/tools-downloads}}。
        \item 安谋科技的Keil $\upmu$Vision,用于编写软件代码(尚未具体研究)。
    \end{enumerate}
\end{frame}

\subsection{现阶段展望}
\begin{frame}
    \frametitle{时间节点}
    \begin{itemize}
        \item 2023/03/15,报名截止。
        \item 2023/05/31,初赛。
        \item 2023/07,分赛区决赛。
        \item 2023/08,全国总决赛。
    \end{itemize}
\end{frame}

\begin{frame}
    \frametitle{时间规划}
    \begin{itemize}
        \item 会后尽快完成杯赛的报名。
        \item 寒假中主要目标是通过一些示例,熟悉一下Verilog和FPGA的使用。
        \item 寒假末开始实验处理器IP的使用。
        \item 之后开始逐步有序的开展工程项目。
        \item 第一步,根据初步的游戏设想,完成游戏机硬件部分的设计。
        \item 第二步,在该平台编写游戏的软件代码。
    \end{itemize}
\end{frame}

\begin{frame}
    \frametitle{难点}
    \begin{enumerate}
        \item 掌握Verilog语言和FPGA的工作流程。
        \item 实现处理器IP的加载和使用。
        \item 实现显示屏和相关外设的驱动。
    \end{enumerate}
\end{frame}

\begin{frame}
    \frametitle{游戏}
    \begin{enumerate}
        \item 游戏的软件代码编写或许是相对最容易的。
        \item 游戏不追求画面,受限于硬件资源(内存在2MB的数量级)。
        \item 游戏的创意很重要,或许可以参考一下JS13K游戏挑战竞赛中的创意。
        \item 或许最重要的比拼点,在于游戏的交互方式,所谓智能游戏机。
    \end{enumerate}
\end{frame}


\end{document}