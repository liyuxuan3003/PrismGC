\section{智能化IP设计}

系统的智能化设计主要体现在输入设备,WII Nunchuck手柄上。WII Nunchuck手柄是由任天堂公司推出的家用游戏主机配套外设。手柄包括2个按钮、2轴摇杆、3轴加速度传感器。手柄基于IIC协议工作,能方便的获取手柄上各个传感器的数据,具体包含:两个按钮(Z按钮和C按钮)是否按下、摇杆的横纵坐标、手柄的三轴加速度。

目前WII手柄仍然在调试中,尚未正式接入游戏机系统中,但已经在Arduino开发板上完成了WII手柄的功能测试,确认了这一解决方案的可行性。测试结果如下
\begin{enumerate}
    \item 手柄的仪器地址为0x52,寄存器地址为0x40。
    \item 手柄在启动时需要向其发送信号0x00。
    \item 手柄发回数据时,接收到的数据为6组8位的二进制数。
    \item 第1,2组数表示手柄摇杆的X轴和Y轴坐标。
    \item 第3,4,5组数表示手柄XYZ轴的加速度数据的高$8$位。
    \item 第6组数中,最低位表示Z按钮是否按下,次低位表示C按钮是否按下,而剩余的$6$位依次代表XYZ轴的加速度数据的低$2$位。
\end{enumerate}

通过WII手柄的加速度数据,可以实现游戏的体感交互,达成智能化游戏机的设计目的。在复赛阶段,项目将重点解决将WII手柄移植到FPGA平台上使用的问题。