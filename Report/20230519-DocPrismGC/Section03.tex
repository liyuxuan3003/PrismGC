\section{软件架构}

\subsection{软硬件功能划分}
系统的软硬件功能划分如下:硬件部分负责实现诸如显示编码、数码管扫描、蜂鸣器驱动等对速度要求较高,需要复杂时序控制的部分。硬件部分将外设封装为若干寄存器,使得软件部分不需要关心驱动外设所需的时序,而是可以直接通过寄存器操作外设的工作状态。软件部分负责游戏功能的实现,根据输入和游戏逻辑控制图像和声音。

\subsection{软件架构}
软件部分主要可以分为两部分:硬件抽象层和软件逻辑层。

硬件抽象层将硬件部分封装的寄存器转换为C语言可用的对象。具体而言,对于每个外设,硬件抽象层会创建一个与外设寄存器排列顺序一致的结构体,随后,硬件抽象层会将外设寄存器所在内存地址强制转换为一个结构体指针,后续编程中就可用通过这个指针操作相应的外设寄存器了。硬件抽象层还会定义一些函数,例如显示部分的图形绘制函数,以及串口通讯部分的读写函数,使软硬件的交互变得更为便利。

软件逻辑层即实现具体游戏逻辑的部分。对于游戏中的每个界面,都会定义一个函数实现界面中的相关功能。在main函数中会调用相应界面的函数,当界面需要切换时,当前界面的函数会返回需要切换界面的编号至main函数,而main函数则会根据这一返回值确定接下来应调用哪一界面的函数。以此实现游戏多个界面间的切换和解耦。

\subsection{软件部分的游戏流程}
游戏部分目前包含两个界面:游戏初始界面、游戏界面。

游戏内容:
玩家通过控制方块进行左右移动以通过正确的区域。别踩黑线!
\begin{enumerate}
    \item 在游戏初始界面时,会播放背景音乐,点击任意按键即开始游戏。
    \item 在进入游戏界面后,显示屏中中会出现一个方块,显示屏下方一条中间有开口的黑线。其中方块为玩家所控制对象,黑线为方块不可通过区域,黑线开口处为方块的可通过区域。同时,数码管会被点亮。其中,左一的数码管的数值表示生命值,左二的数码管的数值表示关卡数,左三及左四的数码管的数值表示游戏得分。
    \item 在开始游戏后,方块将会自动进行向下坠落,玩家需按下键盘上的指定按钮对方块进行左右移动。如果方块从黑线开口处通过,则游戏得分加5,如果方块从黑线所在处通过,则生命值减1,两种情况下会分别播放不同的音符予以提示。
    \item 每次方块从屏幕最上方坠落到最下方的过程为一个周期。在一个周期结束后,方块将会保持上一周期所在的横向位置并再次出现在屏幕最上方,开始新的周期。
    \item 当游戏得分到达50分时,游戏关卡升级,屏幕中将出现两条黑线及其相应的开口区域,开口区域为随机生成,玩家需控制方块从两个开口处依次通过。
    \item 当生命值减少至0时,游戏结束,画面返回至游戏初始页面。
    \item 再次点击任意按键时可再次进入游戏。
\end{enumerate}

游戏目前仅代表一个简单的功能性测试,以检验游戏机系统可以完整运行。后续随着硬件功能的完善,游戏将会相应重新设计,细化游戏内容,丰富游戏交互体验。

